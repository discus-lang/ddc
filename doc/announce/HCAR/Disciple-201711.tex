% Disciple-BD.tex
\begin{hcarentry}{Disciple}
\report{Ben Lippmeier}%11/16
\status{experimental, active development}
\participants{Ben Lippmeier, Jacob Stanley, Chris Hall, Amos Robinson, Ben Sinclair}
\makeheader

The Disciplined Disciple Compiler (DDC) is a research compiler used to
investigate program transformation in the presence of computational effects.
It compiles a family of strict functional core languages and supports region
and effect typing. This extra information provides a handle on the operational
behaviour of code that isn't available in other languages. Programs can be
written in either a pure/functional or effectful/imperative style, and one of
our goals is to provide both styles coherently in the same language.

\WhatsNew

DDC v0.5.1 was released in late October, and is in "working alpha" state.
The main new features are:

\begin{compactitem}
\item Copying garbage collection using the LLVM shadow stack.
\item Implicit parameters, which support Haskell-like ad-hoc overloading using
 dictionaries.
\item Floating point primitives.
\item Travis continuous integration for the GitHub site.
\item A new Sphinx based user guide and homepage.
\end{compactitem}

We are currently working on a new indexed binary format for interface files,
as re-parsing interface files is currently a bottleneck. The file format is to
be provided by the Shimmer project, which has been split out into a separate repo.

\FurtherReading
  \url{http://disciple.ouroborus.net}
  \url{http://disciple.ouroborus.net/release/ddc-0.5.1.html}
  \url{https://github.com/DDCSF/shimmer}

\end{hcarentry}
